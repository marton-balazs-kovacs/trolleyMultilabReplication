% Options for packages loaded elsewhere
\PassOptionsToPackage{unicode}{hyperref}
\PassOptionsToPackage{hyphens}{url}
%
\documentclass[
]{article}
\usepackage{amsmath,amssymb}
\usepackage{lmodern}
\usepackage{ifxetex,ifluatex}
\ifnum 0\ifxetex 1\fi\ifluatex 1\fi=0 % if pdftex
  \usepackage[T1]{fontenc}
  \usepackage[utf8]{inputenc}
  \usepackage{textcomp} % provide euro and other symbols
\else % if luatex or xetex
  \usepackage{unicode-math}
  \defaultfontfeatures{Scale=MatchLowercase}
  \defaultfontfeatures[\rmfamily]{Ligatures=TeX,Scale=1}
\fi
% Use upquote if available, for straight quotes in verbatim environments
\IfFileExists{upquote.sty}{\usepackage{upquote}}{}
\IfFileExists{microtype.sty}{% use microtype if available
  \usepackage[]{microtype}
  \UseMicrotypeSet[protrusion]{basicmath} % disable protrusion for tt fonts
}{}
\makeatletter
\@ifundefined{KOMAClassName}{% if non-KOMA class
  \IfFileExists{parskip.sty}{%
    \usepackage{parskip}
  }{% else
    \setlength{\parindent}{0pt}
    \setlength{\parskip}{6pt plus 2pt minus 1pt}}
}{% if KOMA class
  \KOMAoptions{parskip=half}}
\makeatother
\usepackage{xcolor}
\IfFileExists{xurl.sty}{\usepackage{xurl}}{} % add URL line breaks if available
\IfFileExists{bookmark.sty}{\usepackage{bookmark}}{\usepackage{hyperref}}
\hypersetup{
  pdftitle={Supplementary Analysis},
  hidelinks,
  pdfcreator={LaTeX via pandoc}}
\urlstyle{same} % disable monospaced font for URLs
\usepackage[margin=1in]{geometry}
\usepackage{graphicx}
\makeatletter
\def\maxwidth{\ifdim\Gin@nat@width>\linewidth\linewidth\else\Gin@nat@width\fi}
\def\maxheight{\ifdim\Gin@nat@height>\textheight\textheight\else\Gin@nat@height\fi}
\makeatother
% Scale images if necessary, so that they will not overflow the page
% margins by default, and it is still possible to overwrite the defaults
% using explicit options in \includegraphics[width, height, ...]{}
\setkeys{Gin}{width=\maxwidth,height=\maxheight,keepaspectratio}
% Set default figure placement to htbp
\makeatletter
\def\fps@figure{htbp}
\makeatother
\setlength{\emergencystretch}{3em} % prevent overfull lines
\providecommand{\tightlist}{%
  \setlength{\itemsep}{0pt}\setlength{\parskip}{0pt}}
\setcounter{secnumdepth}{-\maxdimen} % remove section numbering
\usepackage{float} \floatplacement{figure}{H} \newcommand{\beginsupplement}{\setcounter{table}{0}  \renewcommand{\thetable}{S\arabic{table}} \setcounter{figure}{0} \renewcommand{\thefigure}{S\arabic{figure}}}
\usepackage{float}
\usepackage{booktabs}
\usepackage{longtable}
\usepackage{array}
\usepackage{multirow}
\usepackage{wrapfig}
\usepackage{colortbl}
\usepackage{pdflscape}
\usepackage{tabu}
\usepackage{threeparttable}
\usepackage{threeparttablex}
\usepackage[normalem]{ulem}
\usepackage{makecell}
\usepackage{xcolor}
\usepackage{amsmath}
\usepackage{caption}
\ifluatex
  \usepackage{selnolig}  % disable illegal ligatures
\fi

\title{Supplementary Analysis}
\author{}
\date{\vspace{-2.5em}}

\begin{document}
\maketitle

\setcounter{table}{0}  \renewcommand{\thetable}{S\arabic{table}} \setcounter{figure}{0} \renewcommand{\thefigure}{S\arabic{figure}}

\hypertarget{demographics}{%
\section{1. Demographics}\label{demographics}}

\captionsetup[table]{labelformat=empty,skip=1pt}
\begin{longtable}{llrrrrr}
\caption*{
{\large  } \\ 
{\small Table S1. Demographic description of the sample by country.}
} \\ 
\toprule
 & Country & N & Age (SD) & Male \% & Higher education \% & Collectivism\textsuperscript{1} \\ 
\midrule
\multicolumn{1}{l}{By country} \\ 
\midrule
Eastern & China & 1677 & $24.7$ ($7.6$) & $42.7\%$ & $88.5\%$ & $0.075$ \\ 
Eastern & India & 502 & $22.6$ ($6.1$) & $33.7\%$ & $63.7\%$ & $0.069$ \\ 
Eastern & Iran & 235 & $31.0$ ($9.6$) & $55.1\%$ & $89.5\%$ & $0.059$ \\ 
Eastern & Japan & 396 & $44.0$ ($10.8$) & $63.6\%$ & $68.2\%$ & $0.073$ \\ 
Eastern & Lebanon & 20 & $34.1$ ($16.6$) & $50.0\%$ & $95.0\%$ & $0.069$ \\ 
Eastern & Malaysia & 171 & $20.5$ ($2.4$) & $17.0\%$ & $54.4\%$ & $0.133$ \\ 
Eastern & North Macedonia & 282 & $22.1$ ($3.8$) & $55.7\%$ & $26.6\%$ & $0.066$ \\ 
Eastern & Pakistan & 423 & $22.7$ ($3.7$) & $36.4\%$ & $87.5\%$ & $0.078$ \\ 
Eastern & Thailand & 91 & $19.4$ ($1.0$) & $22.0\%$ & $53.8\%$ & $0.079$ \\ 
Eastern & United Arab Emirates & 80 & $24.8$ ($3.7$) & $33.8\%$ & $71.2\%$ & --- \\ 
Southern & Argentina & 253 & $34.3$ ($14.7$) & $27.4\%$ & $79.0\%$ & $0.101$ \\ 
Southern & Chile & 54 & $34.4$ ($13.3$) & $40.7\%$ & $70.4\%$ & $0.079$ \\ 
Southern & Colombia & 278 & $27.8$ ($12.0$) & $41.0\%$ & $87.6\%$ & $0.090$ \\ 
Southern & Czechia & 411 & $28.1$ ($9.6$) & $69.8\%$ & $50.4\%$ & --- \\ 
Southern & Ecuador & 45 & $23.3$ ($4.5$) & $31.1\%$ & $97.7\%$ & $0.130$ \\ 
Southern & France & 935 & $33.8$ ($13.9$) & $17.6\%$ & $71.9\%$ & $0.083$ \\ 
Southern & Hungary & 941 & $21.7$ ($3.8$) & $21.0\%$ & $22.0\%$ & $0.100$ \\ 
Southern & Mexico & 64 & $33.1$ ($5.6$) & $64.1\%$ & $100.0\%$ & $0.082$ \\ 
Southern & Peru & 141 & $24.5$ ($11.4$) & $36.9\%$ & $47.5\%$ & $0.102$ \\ 
Southern & Philippines & 282 & $20.3$ ($3.0$) & $33.7\%$ & $57.0\%$ & $0.127$ \\ 
Southern & Slovakia & 560 & $22.4$ ($6.0$) & $11.6\%$ & $19.7\%$ & --- \\ 
Southern & Turkey & 1369 & $24.6$ ($8.1$) & $24.0\%$ & $42.8\%$ & $0.069$ \\ 
Western & Australia & 1164 & $21.7$ ($6.9$) & $28.2\%$ & $25.2\%$ & $0.032$ \\ 
Western & Austria & 346 & $24.8$ ($8.5$) & $35.1\%$ & $27.8\%$ & --- \\ 
Western & Brazil & 267 & $30.1$ ($12.1$) & $36.7\%$ & $70.4\%$ & $0.097$ \\ 
Western & Bulgaria & 316 & $27.7$ ($11.0$) & $14.9\%$ & $36.4\%$ & $0.066$ \\ 
Western & Canada & 751 & $23.0$ ($7.6$) & $40.9\%$ & $58.9\%$ & $0.029$ \\ 
Western & Croatia & 250 & $21.9$ ($4.1$) & $17.2\%$ & $19.6\%$ & --- \\ 
Western & Denmark & 1299 & $36.7$ ($15.9$) & $46.6\%$ & $57.1\%$ & --- \\ 
Western & Germany & 2887 & $30.2$ ($11.6$) & $29.1\%$ & $21.2\%$ & $0.047$ \\ 
Western & Greece & 515 & $26.1$ ($11.0$) & $20.4\%$ & $52.8\%$ & --- \\ 
Western & Italy & 500 & $35.6$ ($13.9$) & $44.4\%$ & $61.3\%$ & $0.045$ \\ 
Western & Kazakhstan & 122 & $33.5$ ($9.1$) & $29.5\%$ & $100.0\%$ & $0.106$ \\ 
Western & Netherlands & 479 & $20.6$ ($2.8$) & $34.4\%$ & $18.4\%$ & $0.049$ \\ 
Western & New Zealand & 214 & $26.2$ ($10.6$) & $21.5\%$ & $35.0\%$ & $0.032$ \\ 
Western & Poland & 1416 & $30.0$ ($11.0$) & $32.5\%$ & $49.1\%$ & $0.059$ \\ 
Western & Portugal & 716 & $28.5$ ($9.5$) & $36.6\%$ & $70.0\%$ & --- \\ 
Western & Romania & 755 & $24.5$ ($8.5$) & $14.0\%$ & $28.2\%$ & $0.078$ \\ 
Western & Russia & 426 & $31.4$ ($8.2$) & $35.4\%$ & $91.5\%$ & $0.071$ \\ 
Western & Serbia & 485 & $27.1$ ($11.0$) & $25.6\%$ & $52.5\%$ & $0.028$ \\ 
Western & Singapore & 102 & $22.6$ ($1.7$) & $23.5\%$ & $57.8\%$ & $0.030$ \\ 
Western & Spain & 257 & $21.5$ ($6.8$) & $16.0\%$ & $99.2\%$ & $0.041$ \\ 
Western & Switzerland & 549 & $23.0$ ($7.2$) & $29.9\%$ & $22.1\%$ & $0.067$ \\ 
Western & United Kingdom & 865 & $25.2$ ($11.2$) & $23.2\%$ & $42.5\%$ & $0.075$ \\ 
Western & United States & 3611 & $20.7$ ($4.9$) & $23.4\%$ & $22.9\%$ & $0.000$ \\ 
\midrule
\multicolumn{1}{l}{By region} \\ 
\midrule
Eastern & --- & 3877 & $26.1$ ($9.7$) & $42.9\%$ & $75.2\%$ & --- \\ 
Southern & --- & 5333 & $26.3$ ($10.5$) & $27.2\%$ & $48.7\%$ & --- \\ 
Western & --- & 18292 & $25.9$ ($10.4$) & $28.4\%$ & $38.4\%$ & --- \\ 
\midrule
\multicolumn{1}{l}{All} \\ 
All & --- & 27502 & $26.0$ ($10.3$) & $30.3\%$ & $45.8\%$ & --- \\ 
 \bottomrule
\end{longtable}
\vspace{-5mm}
\begin{minipage}{\linewidth}
\textsuperscript{1}Distance from the US in collectivism. Some countries do not have a collectivism score. \\ 
\end{minipage}

\hypertarget{additional-analysis}{%
\section{2. Additional analysis}\label{additional-analysis}}

\hypertarget{effect-of-physical-contact}{%
\subsection{Effect of physical
contact}\label{effect-of-physical-contact}}

In sum, when assassing the effect of physical force, we found
inconclusive evidence for the effect of physical contact, regardless of
dilemma type (trolley/speedboat). The summary of the results can be
found in the tables below.

\begin{table}[H]

\caption{\label{tab:physical contact }Table S2. The effect of phyisical contact on moral dilemma judgements on Trolley dilemmas}
\resizebox{\linewidth}{!}{
\begin{tabular}[t]{lllrrrrrl}
\toprule
\textbf{Exclusion} & \textbf{Cluster} & \textbf{BF} & \textbf{t} & \textbf{df} & \textbf{p} & \textbf{Cohen's d} & \textbf{Raw effect} & \textbf{$89\%$ CI}\\
\midrule
 & Eastern & 4.11e-01 & -0.22 & 121.63 & 0.828 & 0.03 & 0.07 & {}[-0.36, 0.44]\\

 & Southern & 1.94e-01 & 0.46 & 388.70 & 0.643 & 0.04 & -0.09 & {}[-0.38, 0.18]\\

\multirow[t]{-3}{*}{\raggedright\arraybackslash Exclusion} & Western & 1.72e-01 & 0.18 & 512.54 & 0.857 & 0.01 & -0.02 & {}[-0.23, 0.18]\\
\cmidrule{1-9}
 & Eastern & 1.76e-01 & 0.61 & 254.06 & 0.544 & 0.06 & -0.12 & {}[-0.37, 0.21]\\

 & Southern & 9.49e-02 & 1.17 & 756.54 & 0.244 & 0.08 & -0.15 & {}[-0.34, 0.08]\\

\multirow[t]{-3}{*}{\raggedright\arraybackslash Include familiar} & Western & 7.91e-02 & 0.73 & 1099.31 & 0.464 & 0.03 & -0.07 & {}[-0.2, 0.08]\\
\bottomrule
\end{tabular}}
\end{table}

\hypertarget{comparing-the-standard-switch-and-standard-footbridge-dilemmas}{%
\subsection{Comparing the standard switch and standard footbridge
dilemmas}\label{comparing-the-standard-switch-and-standard-footbridge-dilemmas}}

When comparing the standard switch and standard footbridge dilemmas in
all clusters for the trolley and the speedboat tasks we found evidence
for a difference between the two dilemmas in moral acceptability
ratings. The summary results of each comparison separately can be found
in Tables below.

\begin{table}[H]

\caption{\label{tab:standard switch and footbridge}Table S3. Comparing the Standard Switch and Standard Footbridge Dilemmas (all exclusion applied).}
\begin{tabular}[t]{llrlrl}
\toprule
\textbf{Dilemma} & \textbf{Cluster} & \textbf{t} & \textbf{Bf} & \textbf{df} & \textbf{p}\\
\midrule
 & Eastern & 4.81 & 2.49e+03 & 154.32 & < .001\\

 & Southern & 10.38 & 2.32e+19 & 229.92 & < .001\\

\multirow[t]{-3}{*}{\raggedright\arraybackslash Trolley} & Western & 16.88 & 1.99e+54 & 780.55 & < .001\\
\cmidrule{1-6}
 & Eastern & 6.29 & 3.61e+05 & 130.68 & < .001\\

 & Southern & 9.61 & 5.50e+15 & 335.65 & < .001\\

\multirow[t]{-3}{*}{\raggedright\arraybackslash Speedboat} & Western & 14.58 & 4.01e+41 & 1618.57 & < .001\\
\bottomrule
\end{tabular}
\end{table}

\begin{table}[H]

\caption{\label{tab:switch vs footbridge no familiarity}Table S4. Comparing the Standard Switch and Standard Footbridge Dilemmas (familiarity exclusion not applied)}
\begin{tabular}[t]{llrlrl}
\toprule
\textbf{Dilemma} & \textbf{Cluster} & \textbf{t} & \textbf{Bf} & \textbf{df} & \textbf{p}\\
\midrule
 & Eastern & 6.79 & 8.73e+07 & 282.63 & < .001\\

 & Southern & 14.59 & 3.91e+38 & 544.52 & < .001\\

\multirow[t]{-3}{*}{\raggedright\arraybackslash Trolley} & Western & 27.95 & 1.86e+148 & 2310.31 & < .001\\
\cmidrule{1-6}
 & Eastern & 7.30 & 8.06e+08 & 257.05 & < .001\\

 & Southern & 11.03 & 6.30e+22 & 647.03 & < .001\\

\multirow[t]{-3}{*}{\raggedright\arraybackslash Speedboat} & Western & 24.06 & 4.93e+116 & 4352.64 & < .001\\
\bottomrule
\end{tabular}
\end{table}

\#\#Analysing familiar participants

As we registered, we conducted the analysis on familiar participants,
the results can be found below.

\begin{table}[H]

\caption{\label{tab:Study1ab Bayesian}Table S5. The effect of personal force on moral dilemma judgements (familiar participants).}
\begin{tabular}[t]{lllrrlrrl}
\toprule
\textbf{Dilemma} & \textbf{Cluster} & \textbf{BF} & \textbf{t} & \textbf{df} & \textbf{p} & \textbf{Cohen's d} & \textbf{Raw effect} & \textbf{$89\%$ CI}\\
\midrule
 & Eastern & 1.65e+02 & -3.65 & 437.72 & <.001 & 0.35 & 0.73 & {}[0.34, 0.98]\\

 & Southern & 1.76e+05 & -5.35 & 721.33 & <.001 & 0.40 & 0.82 & {}[0.53, 1.03]\\

\multirow[t]{-3}{*}{\raggedright\arraybackslash Trolley} & Western & 2.12e+03 & -4.34 & 778.76 & <.001 & 0.31 & 0.67 & {}[0.38, 0.88]\\
\cmidrule{1-9}
 & Eastern & 2.06e+00 & -1.82 & 383.27 & 0.07 & 0.18 & 0.37 & {}[0, 0.62]\\

 & Southern & 3.2e+03 & -4.32 & 469.42 & <.001 & 0.35 & 0.68 & {}[0.4, 0.88]\\

\multirow[t]{-3}{*}{\raggedright\arraybackslash Speedboat} & Western & 5.4e+05 & -5.56 & 707.92 & <.001 & 0.40 & 0.81 & {}[0.52, 0.99]\\
\bottomrule
\end{tabular}
\end{table}

\begin{table}[H]

\caption{\label{tab:Study2ab Bayesian}Table S6. The interaction of personal force and intention on moral dilemma judgemnts (familiar participants).}
\begin{tabular}[t]{lllrllrr}
\toprule
\textbf{Dilemma} & \textbf{Cluster} & \textbf{BF} & \textbf{F} & \textbf{df} & \textbf{p} & \textbf{Partial $\eta^2$} & \textbf{Raw effect}\\
\midrule
 & Eastern & 5.57e+01 & -0.500 & {}[-0.67, -0.2] & 0.001 & 0.043 & -2.00\\

 & Southern & 6.79e+05 & -0.463 & {}[-0.56, -0.31] & <.001 & 0.047 & -1.85\\

\multirow[t]{-3}{*}{\raggedright\arraybackslash Trolley} & Western & 4.7e+18 & -0.288 & {}[-0.33, -0.24] & <.001 & 0.019 & -1.15\\
\cmidrule{1-8}
 & Eastern & 7.37e-01 & -0.001 & {}[-0.22, 0.21] & 0.993 & 0.000 & -0.01\\

 & Southern & 2.35e+01 & -0.355 & {}[-0.48, -0.1] & 0.003 & 0.016 & -1.42\\

\multirow[t]{-3}{*}{\raggedright\arraybackslash Speedboat} & Western & 2.26e+04 & -0.143 & {}[-0.18, -0.09] & <.001 & 0.005 & -0.57\\
\bottomrule
\end{tabular}
\end{table}

\hypertarget{oxford-utilitarianism-scale}{%
\subsection{Oxford utilitarianism
Scale}\label{oxford-utilitarianism-scale}}

As we registered, we first plot statistics of the Oxford Utilitarianism
Scale in each cultural clusters. We applied no exclusion criteria during
this analysis. Note however, that due to a technical mistake, some
hungarian participants did not see one of the items in the OUS, hence,
they were excluded from this analysis.

\begin{figure}[H]
\includegraphics[width=1\linewidth]{supplementary_materials_files/figure-latex/unnamed-chunk-2-1} \caption{Results on the Oxford Utilitarianism Scale by regions. The X axis shows the two subscales, while the Y axis shows the ratings. The center of the boxplots are the median, while the lower bound corresponds to the first quartiles (25th percentile), and the upper bound corresponds to the third quartile (75th percentile). The lower whisker represents the minimum value at no more than 1.5 times the inter quartile range from the lower bound, while the upper whisker respresents the maximum value at no further than 1.5 times the inter quartile range fro the upper bound.}\label{fig:unnamed-chunk-2}
\end{figure}

As registered, we also report means and confidence intervals for each
cultural cluster and each subscale of the Oxford Utilitarianism Scale.

\begin{table}[H]

\caption{\label{tab:OUS means + CI}Table S7. Means and confidence intervals of the Oxford Utilitarianism Scale.)}
\begin{tabular}[t]{lrlrl}
\toprule
\multicolumn{1}{c}{ } & \multicolumn{2}{c}{Instrumental Harm} & \multicolumn{2}{c}{Impartial Beneficence} \\
\cmidrule(l{3pt}r{3pt}){2-3} \cmidrule(l{3pt}r{3pt}){4-5}
\textbf{Cluster} & \textbf{Mean} & \textbf{$95\%$ CI} & \textbf{Mean} & \textbf{$95\%$ CI}\\
\midrule
Eastern & 3.404114 & {}[3.44,3.37] & 3.701470 & {}[3.74,3.67]\\
Southern & 3.339469 & {}[3.37,3.3] & 4.219580 & {}[4.25,4.19]\\
Western & 3.500703 & {}[3.52,3.48] & 4.013701 & {}[4.03,4]\\
\bottomrule
\end{tabular}
\end{table}

We also reported correlations between each Oxford Utilitarianism Scale
subscales and moral acceptability ratings on each moral dilemma. Results
suggests a higher correlation between acceptability ratings and the
Instrumental Harm scale, anda somewhat lower correlation between
Impartial Beneficence and acceptability ratings.

\begin{table}[H]

\caption{\label{tab:OUS correlations}Table S8. Correlational analysis of the Oxford Utilitarianism Scale subscales with moral accaptability ratings on moral dilemmas.)}
\begin{tabular}[t]{llrlrll}
\toprule
\multicolumn{1}{c}{ } & \multicolumn{2}{c}{Impartial Beneficence} & \multicolumn{2}{c}{Instrumental Harm} & \multicolumn{1}{c}{ } \\
\cmidrule(l{3pt}r{3pt}){2-3} \cmidrule(l{3pt}r{3pt}){4-5}
\textbf{Dilemma} & \textbf{Cluster} & \textbf{r} & \textbf{p} & \textbf{r} & \textbf{p} & \textbf{df}\\
\midrule
 & Eastern & 0.45 & <.001 & 0.16 & <.001 & 3875\\

 & Southern & 0.44 & <.001 & 0.05 & <.001 & 4860\\

\multirow[t]{-3}{*}{\raggedright\arraybackslash Trolley} & Western & 0.42 & <.001 & 0.08 & <.001 & 17077\\
\cmidrule{1-7}
 & Eastern & 0.41 & <.001 & 0.20 & <.001 & 3875\\

 & Southern & 0.40 & <.001 & 0.08 & <.001 & 4860\\

\multirow[t]{-3}{*}{\raggedright\arraybackslash Speedboat} & Western & 0.40 & <.001 & 0.12 & <.001 & 17077\\
\bottomrule
\end{tabular}
\end{table}

\#Exploratory analysis on overall utilitarianism and collectivism

Although not part of the planned analysis, we hypothesized that
country-level collectivism would be negatively associated with
utilitarian responding (i.e., higher morall acceptibility ratings). We
found no evidence for this hypothesis, regardless of familiarity
exclusion or dilemma context. Interestingly, however, we found strong
evidence for the association between vertical individualism and average
moral acceptibility ratings on moral dilemmas, regardless of dilemma
context or exclusion criteria. The positive association means that
higher levels of vertical individualism is associated with higher
acceptance of the utilitarian response option. Although we hypothesized
that it would be collectivism that makes people \emph{more} emotional
and therefore, less utilitarian, we speculate that individualism made
people \emph{less} emotional and therefore, more utilitarian.

In all of the regression models below, we added the random intercept of
countries.

\begin{table}[H]

\caption{\label{tab:analysis individual + country level study 2a}Table S9. Is the interaction of personal force and intention affected by individualism/collectivism on Trolley dilemmas?}
\begin{tabular}[t]{lrrlrrl}
\toprule
\multicolumn{1}{c}{ } & \multicolumn{3}{c}{With familiarity exclusion} & \multicolumn{3}{c}{No familiarity exclusion} \\
\cmidrule(l{3pt}r{3pt}){2-4} \cmidrule(l{3pt}r{3pt}){5-7}
\textbf{Variable} & \textbf{BF} & \textbf{b} & \textbf{p} & \textbf{BF} & \textbf{b} & \textbf{p}\\
\midrule
Country-level collectivism & 2.5e-01 & -2.76 & 0.409 & 2.5e-01 & -3.82 & 0.216\\
H. Collectivism & 8.0e-02 & -0.02 & 0.491 & 4.0e-02 & 0.00 & 0.861\\
H. Individualism & 2.8e+00 & 0.06 & 0.005 & 2.1e+01 & 0.05 & <.001\\
V. Collectivism & 1.6e-01 & 0.03 & 0.135 & 4.0e-02 & -0.01 & 0.679\\
V. Individualism & 3.6e+13 & 0.15 & <.001 & 1.9e+23 & 0.12 & <.001\\
\bottomrule
\end{tabular}
\end{table}

\begin{table}[H]

\caption{\label{tab:analysis individual + country level study 2b}Table S10. Is the interaction of personal force and intention affected by individualism/collectivism on Speedboat dilemmas?}
\begin{tabular}[t]{lrrlrrl}
\toprule
\multicolumn{1}{c}{ } & \multicolumn{3}{c}{With familiarity exclusion} & \multicolumn{3}{c}{No familiarity exclusion} \\
\cmidrule(l{3pt}r{3pt}){2-4} \cmidrule(l{3pt}r{3pt}){5-7}
\textbf{Variable} & \textbf{BF} & \textbf{b} & \textbf{p} & \textbf{BF} & \textbf{b} & \textbf{p}\\
\midrule
Country-level collectivism & 6.4e-01 & -6.50 & 0.048 & 4.1e-01 & -5.25 & 0.041\\
H. Collectivism & 7.0e-02 & -0.01 & 0.62 & 4.0e-02 & 0.00 & 0.756\\
H. Individualism & 6.0e-02 & 0.00 & 0.876 & 6.0e-02 & 0.01 & 0.335\\
V. Collectivism & 8.0e-02 & 0.02 & 0.423 & 8.0e-02 & -0.01 & 0.253\\
V. Individualism & 6.2e+09 & 0.13 & <.001 & 1.1e+17 & 0.10 & <.001\\
\bottomrule
\end{tabular}
\end{table}

\#\#With exclusions

\begin{figure}[H]
\includegraphics[width=0.8\linewidth]{supplementary_materials_files/figure-latex/study2a overall individualism plot-1} \caption{Correlation between country-level individualism/collectivism and moral accessibility ratings on the Trolley dilemmas (higher moral acceptibility means higher acceptibility of the utilitarian choice).}\label{fig:study2a overall individualism plot}
\end{figure}

\begin{figure}
\includegraphics[width=0.8\linewidth]{supplementary_materials_files/figure-latex/study2a vertical horizontal individualism-1} \caption{Personal level individualism/collectivism effects on moral acceptibility ratings (trolley dilemmas)}\label{fig:study2a vertical horizontal individualism}
\end{figure}

\begin{figure}[H]
\includegraphics[width=0.8\linewidth]{supplementary_materials_files/figure-latex/study2b overall individualism plot-1} \caption{Correlation between country-level individualism/collectivism and moral accessibility ratings on the Speedboat dilemmas (higher moral acceptibility means higher acceptibility of the utilitarian choice)}\label{fig:study2b overall individualism plot}
\end{figure}

\begin{figure}
\includegraphics[width=0.8\linewidth]{supplementary_materials_files/figure-latex/study2b vertical horizontal individualism-1} \caption{Personal level individualism/collectivism effects on moral acceptibility ratings (speedboat dilemmas)}\label{fig:study2b vertical horizontal individualism}
\end{figure}

\#\#Without familiarity exclusion

\begin{figure}[H]
\includegraphics[width=0.8\linewidth]{supplementary_materials_files/figure-latex/study2a overall individualism plot w/o exc-1} \caption{Correlation between country-level individualism/collectivism and moral accessibility ratings on the Trolley dilemmas (higher moral acceptibility means higher acceptibility of the utilitarian choice)}\label{fig:study2a overall individualism plot w/o exc}
\end{figure}

\begin{figure}
\includegraphics[width=0.8\linewidth]{supplementary_materials_files/figure-latex/study2a vertical horizontal individualism w/o excl-1} \caption{Personal level individualism/collectivism effects on moral acceptibility ratings (trolley dilemmas)}\label{fig:study2a vertical horizontal individualism w/o excl}
\end{figure}

\begin{figure}[H]
\includegraphics[width=0.8\linewidth]{supplementary_materials_files/figure-latex/study2b overall individualism plot w/o excl-1} \caption{Correlation between country-level individualism/collectivism and moral accessibility ratings on the Speedboat dilemmas (higher moral acceptibility means higher acceptibility of the utilitarian choice)}\label{fig:study2b overall individualism plot w/o excl}
\end{figure}

\begin{figure}
\includegraphics[width=0.8\linewidth]{supplementary_materials_files/figure-latex/study2b vertical horizontal individualism w/o excl-1} \caption{Personal level individualism/collectivism effects on moral acceptibility ratings (speedboat dilemmas)}\label{fig:study2b vertical horizontal individualism w/o excl}
\end{figure}

\end{document}
